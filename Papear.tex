% !TeX spellcheck = en_GB
% !TeX program = lualatex
%
% v 2.3  Feb 2019   Volker RW Schaa
%		# changes in the collaboration therefore updated file "jacow-collaboration.tex"
%		# all References with DOIs have their period/full stop before the DOI (after pp. or year)
%		# in the author/affiliation block all ZIP codes in square brackets removed as it was not %         understood as optional parameter and ZIP codes had bin put in brackets
%       # References to the current IPAC are changed to "IPAC'19, Melbourne, Australia"
%       # font for ‘url’ style changed to ‘newtxtt’ as it is easier to distinguish "O" and "0"
%
\documentclass[a4paper,
               %boxit,        % check whether paper is inside correct margins
               %titlepage,    % separate title page
               %refpage       % separate references
               %biblatex,     % biblatex is used
               %keeplastbox,   % flushend option: not to un-indent last line in References
               %nospread,     % flushend option: do not fill with whitespace to balance columns
               %hyphens,      % allow \url to hyphenate at "-" (hyphens)
               %xetex,        % use XeLaTeX to process the file
               %luatex,       % use LuaLaTeX to process the file
               ]{jacow}
%
% ONLY FOR \footnote in table/tabular
%
\usepackage{pdfpages,multirow,ragged2e} %
%
% CHANGE SEQUENCE OF GRAPHICS EXTENSION TO BE EMBEDDED
% ----------------------------------------------------
% test for XeTeX where the sequence is by default eps-> pdf, jpg, png, pdf, ...
%    and the JACoW template provides JACpic2v3.eps and JACpic2v3.jpg which
%    might generates errors, therefore PNG and JPG first
%
\makeatletter%
	\ifboolexpr{bool{xetex}}
	 {\renewcommand{\Gin@extensions}{.pdf,%
	                    .png,.jpg,.bmp,.pict,.tif,.psd,.mac,.sga,.tga,.gif,%
	                    .eps,.ps,%
	                    }}{}
\makeatother

% CHECK FOR XeTeX/LuaTeX BEFORE DEFINING AN INPUT ENCODING
% --------------------------------------------------------
%   utf8  is default for XeTeX/LuaTeX
%   utf8  in LaTeX only realises a small portion of codes
%
\ifboolexpr{bool{xetex} or bool{luatex}} % test for XeTeX/LuaTeX
 {}                                      % input encoding is utf8 by default
 {\usepackage[utf8]{inputenc}}           % switch to utf8

\usepackage[USenglish]{babel}
%\usepackage[spanish]{babel}
\newenvironment{keywords}
{\par\small\textbf{Keywords}}
{\par}
%
% if BibLaTeX is used
%
\ifboolexpr{bool{jacowbiblatex}}%
 {%
  \addbibresource{jacow-test.bib}
  \addbibresource{biblatex-examples.bib}
 }{}
\listfiles

%%
%%   Lengths for the spaces in the title
%%   \setlength\titleblockstartskip{..}  %before title, default 3pt
%%   \setlength\titleblockmiddleskip{..} %between title + author, default 1em
%%   \setlength\titleblockendskip{..}    %afterauthor, default 1em

\begin{document}

\title{Learning Enriched Features for Fast Image Restoration and Enhancement\thanks{Work supported by ...}}

\author{Fredy Abel Huanca Torres\thanks{Universidad Nacional San Agustín de Arequipa, fhuancat@unsa.edu.pe},
		Jose Edison Perez Mamani\thanks{Universidad Nacional San Agustín de Arequipa, jperezma@unsa.edu.pe} y
		Henrry Ivan Arias Mamani\thanks{Universidad Nacional San Agustín de Arequipa, hariasm@unsa.edu.pe}}
	
\maketitle

%
\begin{abstract}
Image restoration is an important task in surveillance, computational photography, medical imaging, and remote sensing. Recently, convolutional neural networks (CNNs) have achieved dramatic improvements over conventional approaches. This paper presents a novel architecture with the collective goals of maintaining spatially precise high-resolution representations and receiving contextual solid information from the low-resolution representations. The core of the approach is a multi-scale residual block containing several key elements: parallel multi-resolution convolution streams, information exchange across the multi-resolution streams, spatial and channel attention mechanisms, and attention-based multi-scale feature aggregation. Extensive experiments on real image benchmark datasets demonstrate that MIRNet achieves state-of-the-art results for various image processing tasks.
\end{abstract}

\begin{keywords}
Image denoising, super-resolution, and image enhancement
\end{keywords}

\section{Introducción}

La restauración de imágenes es una tarea importante en vigilancia, fotografía computacional, imágenes médicas y teledetección. Recientemente, las redes neuronales convolucionales (CNN) han logrado mejoras dramáticas sobre los enfoques convencionales. Este artículo presenta una arquitectura novedosa con los objetivos colectivos de mantener representaciones espacialmente precisas de alta resolución y recibir información contextual sólida de las representaciones de baja resolución. El núcleo del enfoque es un bloque residual multiescala que contiene varios elementos clave: flujos paralelos de convolución multiresolución, intercambio de información a través de los flujos multiresolución, mecanismos de atención espacial y de canales, y agregación de características multiescala basada en la atención. Extensos experimentos en conjuntos de datos de referencia de imágenes reales demuestran que MIRNet logra resultados de vanguardia para una variedad de tareas de procesamiento de imágenes.

\begin{figure*}[!h]
    \centering
    \includegraphics*[width=.9\textwidth]{entrenamiento}
    \caption{Ejemplo de ejecución del entrenamiento.}
    \label{fig:ejem_Entrenamiento}
\end{figure*}

\section{Trabajo Relacionados}
 los autores revisan los métodos más relevantes para el procesamiento de imágenes de bajo nivel, incluyendo la eliminación de ruido, la superresolución y la mejora de imagen. A continuación, se resumen algunos de los métodos mencionados:

- Para la eliminación de ruido, como los basados en transformaciones de coeficientes y promedio de píxeles vecinos \cite{zhang2017beyond}. Además, se han propuesto enfoques basados en redes neuronales convolucionales profundas para esta tarea, como DnCNN \cite{zhang2017beyond} y RED30 \cite{mao2019red30}. Estos métodos utilizan arquitecturas profundas para aprender a eliminar el ruido de las imágenes a partir del conjunto de entrenamiento y generar imágenes limpias a partir de nuevas imágenes o imágenes dañadas por el ruido. Los resultados experimentales muestran que estos métodos pueden superar a los métodos clásicos en términos tanto visuales como cuantitativos.
- Para la superresolución, algunos de los métodos mencionados son bicubic interpolation, que es el método más comúnmente utilizado para generar imágenes de alta resolución a partir de imágenes de baja resolución, y enfoques basados en redes neuronales convolucionales profundas como VDSR \cite{vdsr}, DRCN \cite{kim2016deeply} y SRCNN \cite{dong2016image}. Estos métodos utilizan arquitecturas profundas para aprender a generar imágenes de alta resolución a partir del conjunto de entrenamiento y mejorar la calidad visual y cuantitativa de las imágenes generadas. Los resultados experimentales muestran que estos métodos pueden superar a los métodos clásicos como bicubic interpolation en términos tanto visuales como cuantitativos.
- Para la mejora de imagen,  algunos de los métodos mencionados son MemNet \cite{tai2017memnet} y FFDNet \cite{zhang2018ffdnet}, que son enfoques basados en redes neuronales convolucionales profundas que procesan características a su resolución original y fusionan información contextual de múltiples ramas paralelas. Además, se menciona el conjunto de datos MIT-Adobe FiveK \cite{dabov2007image}, que contiene imágenes de diversas escenas interiores y exteriores capturadas con cámaras DSLR en diferentes condiciones de iluminación, y se han utilizado las imágenes mejoradas por expertos como referencia para evaluar el rendimiento de los algoritmos. Los resultados experimentales muestran que los métodos basados en redes neuronales convolucionales profundas pueden mejorar significativamente la calidad visual y cuantitativa de las imágenes en comparación con los métodos clásicos.

Los autores también destacan que muchos de estos métodos están diseñados para una tarea específica y no son fácilmente adaptables a otras tareas. Por lo tanto, proponen MIRNet como un modelo unificado que puede manejar múltiples tareas de procesamiento de imágenes de bajo nivel con un rendimiento estatal del arte.

\section{Método propuesto}
La restauración y mejora de imágenes es un problema importante en muchas aplicaciones prácticas, como la fotografía digital, la medicina y la vigilancia por video \cite{zhang2017beyond}. En este articulo, se presenta una nueva técnica basada en redes neuronales convolucionales (CNN) para abordar este problema, en la figura ~\ref{fig:ejem_Entrenamiento}. El método propuesto se llama MIRNet (Red Mejorada de Restauración de Imágenes) y utiliza bloques residuales de múltiples escalas para mantener las características de alta resolución a lo largo de la jerarquía de la red \cite{tian2020learning}.


La clave del éxito del MIRNet es su capacidad para separar el contenido no deseado degradado del verdadero contenido espacialmente detallado. Esto se logra mediante el uso de grandes contextos que amplían el campo receptivo. Sin embargo, esto puede resultar en una pérdida de detalles espaciales precisos. Para abordar este problema, los autores proponen una nueva técnica que mantiene las características originales de alta resolución a lo largo de la jerarquía de la red \cite{tian2020learning}.


El MIRNet también utiliza un mecanismo llamado "atención" para enfocarse en las regiones más importantes y reducir el ruido en las regiones menos importantes. Este mecanismo ayuda a mejorar aún más la calidad visual y perceptual del resultado final. Además, el MIRNet es capaz de manejar diferentes tipos de distorsiones, como el ruido, la borrosidad y la falta de detalles \cite{tian2020learning}.

Los experimentos realizados en este articulo demuestran que el MIRNet supera a otros métodos de restauración de imágenes en términos de calidad visual y perceptual. Además, el MIRNet es capaz de restaurar imágenes con una mayor velocidad y eficiencia que otros métodos \cite{tian2020learning}.

%\begin{figure}[!htb]
%   \centering
%   \includegraphics*[width=.7\columnwidth]{JACpic_mc}
%   \caption{Layout of papers.}
%   \label{fig:paper_layout}
%\end{figure}

%\begin{figure*}[!tbh]
%    \centering
%    \includegraphics*[width=\textwidth]{JACpic2}
%
%    \caption{Example of a full-width figure showing the JACoW Team at their annual
%    	     meeting in December 2018. This figure has a multi-line caption that has to be
%    	     justified rather than centred.}
%    \label{fig:jacow_team}
%\end{figure*}

\begin{figure}[!h]
    \centering
    \includegraphics*[width=.5\textwidth]{ll-1}
    \includegraphics*[width=.5\textwidth]{ll-2}
    \includegraphics*[width=.5\textwidth]{ll-3}
    \includegraphics*[width=.5\textwidth]{ll-4}
    \includegraphics*[width=.5\textwidth]{ll-5}
    \includegraphics*[width=.5\textwidth]{ll-6}
    \caption{Ejemplo de ejecución de una imagen original hasta la imagen optimizada.}
    \label{fig:ejem_procesamiento}
\end{figure}

\section{EXPERIMENTO}
Los autores describen los detalles del entrenamiento y evaluación del modelo MIRNet para tres tareas de procesamiento de imágenes de bajo nivel: eliminación de ruido, superresolución e imagen mejorada. Para cada tarea, utilizaron cinco conjuntos de datos reales diferentes y compararon el rendimiento de MIRNet con otros métodos del estado del arte.

Para la tarea de eliminación de ruido, se utilizó el conjunto de datos DnD \cite{dnd}, que consta de 1,800 pares de imágenes con ruido y sin ruido. El modelo se entrenó utilizando el optimizador Adam durante 200 épocas con una tasa de aprendizaje inicial de $2x10^{-4}$. También se realizó un análisis ablativo para evaluar el impacto individual de cada componente arquitectónico en el rendimiento final del modelo.

Para la tarea de superresolución, se utilizaron cuatro conjuntos diferentes: Set5, Set14, BSD100 y Urban100. El modelo se entrenó utilizando el optimizador Adam durante 400 épocas con una tasa de aprendizaje inicial de $2x10^{-4}$. También se realizó un análisis ablativo para evaluar el impacto individual del tamaño del parche y la profundidad en el rendimiento final del modelo.

Para la tarea de mejora de imagen, se utilizaron tres conjuntos diferentes: PIRM2018-SR-track2-validation, PIRM2018-SR-track2-test y DIV2K. El modelo se entrenó utilizando el optimizador Adam durante 800 épocas con una tasa de aprendizaje inicial de$2x10^{-4}$.

En general, los resultados experimentales muestran que MIRNet supera a otros métodos estatales del arte en las tres tareas de procesamiento de imágenes de bajo nivel. Además, el análisis ablativo revela que cada componente arquitectónico del modelo contribuye significativamente al rendimiento final.

En la figura ~\ref{fig:ejem_procesamiento}, se describe el proceso de experimentación del método propuesto para el procesamiento de imágenes utilizando una red neuronal profunda llamada MIRNet. Se entrenó la red neuronal utilizando un conjunto de datos de entrenamiento y se evaluó en cinco conjuntos de datos diferentes para tareas como denoising, super-resolución e imagen mejorada. Durante el entrenamiento, se utilizaron parches de tamaño 128x128 y operaciones de aumento de datos para mejorar la precisión del modelo. La tasa de aprendizaje disminuyó gradualmente durante el entrenamiento para mejorar la estabilidad del modelo. Los resultados muestran que el método propuesto supera a los métodos existentes en todos los conjuntos de datos evaluados y demuestra una buena capacidad generalización a través de diferentes conjuntos de datos.
%\begin{figure}[!htb]
%   \centering
%   \includegraphics*[width=1\columnwidth]{ll-1}
%   \caption{Layout of papers.}
%   \label{fig:paper_layout}
%\end{figure}



\section{CONCLUSION}

En este estudio, se realizó una revisión exhaustiva de los métodos de aprendizaje profundo para la eliminación de ruido en imágenes. Se discutieron los enfoques más populares y se compararon sus fortalezas y debilidades.

Se concluyó que los métodos basados en redes neuronales convolucionales (CNN) son los más efectivos para la eliminación de ruido en imágenes \cite{zhang2017beyond, lehtinen2018noise2noise, tai2017image}. Además, se encontró que el uso de arquitecturas profundas y técnicas de entrenamiento avanzadas, como la normalización por lotes y la regularización, puede mejorar significativamente el rendimiento del modelo \cite{lefkimmiatis2018universal}.

También se destacó la importancia del conjunto de datos utilizado para entrenar y evaluar los modelos. Se recomendó el uso de conjuntos de datos grandes y diversos para garantizar que los modelos sean capaces de generalizar bien a diferentes tipos de ruido y condiciones \cite{zhang2018learning}.

En general, se concluyó que el aprendizaje profundo ha demostrado ser una herramienta poderosa para la eliminación de ruido en imágenes y que hay muchas oportunidades para futuras investigaciones en esta área.


%
% only for "biblatex"
%
\ifboolexpr{bool{jacowbiblatex}}%
	{\printbibliography}%
	{
	\begin{thebibliography}{9} % Use for 1-9 references
	
	   \bibitem{tian2020learning}
        Y. Tian, Y. Zhang, Y. Fu, and B. Ghanem, "Learning from scratch for low-level vision," \textit{arXiv preprint arXiv:2003.06792}, 2020.

	
        \bibitem{tai2017memnet}
        Y. Tai, J. Yang and X. Liu, "MemNet: A persistent memory network for image restoration," in \emph{Proceedings of the IEEE International Conference on Computer Vision}, Venice, Italy, Oct.-Dec., 2017, pp. 4549-4557.

        \bibitem{zhang2018ffdnet}
        K. Zhang, W. Zuo and L. Zhang, "FFDNet: Toward a fast and flexible solution for CNN-based image denoising," \emph{IEEE Transactions on Image Processing}, vol. 27, no. 9, pp. 4608-4622, Sep., 2018.


        \bibitem{mao2019red30}
        Mao, X., Shen, C.,   Yang, Y.-B. (2019). RED30: A Deep Residual Encoder-Decoder Network for Image Restoration and Enhancement. IEEE Transactions on Image Processing, 28(12), 6237–6252.

        \bibitem{vdsr}
        Jiwon Kim, Jung Kwon Lee, and Kyoung Mu Lee.
        Accurate image super-resolution using very deep convolutional networks.
        {\em IEEE Transactions on Pattern Analysis and Machine Intelligence}, 38(2):295--307, 2016.

        \bibitem{dong2016image}
        Chao Dong, Chen Change Loy, Kaiming He, and Xiaoou Tang.
        Image super-resolution using deep convolutional networks.
        {\em IEEE Transactions on Pattern Analysis and Machine Intelligence}, 38(2):295--307, 2016.

        \bibitem{kim2016deeply}
        Jiwon Kim, Jung Kwon Lee, and Kyoung Mu Lee.
        \newblock Deeply-recursive convolutional network for image super-resolution.
        \newblock In {\em Proceedings of the IEEE Conference on Computer Vision and Pattern Recognition (CVPR)}, pages 1637--1645, 2016.
	
        \bibitem{dabov2007image}
        Kostadin Dabov, Alessandro Foi, Vladimir Katkovnik, and Karen Egiazarian.
        \newblock Image denoising using sparse 3d transform-domain collaborative filtering.
        \newblock {\em IEEE Transactions on Image Processing}, 16(8):2080--2095, 2007.

        \bibitem{lehtinen2018noise2noise}
        J. Lehtinen, J. Munkberg, J. Hasselgren, S. Laine, T. Karras, M. Aittala, and T. Aila,
        ''Noise2Noise: Learning Image Restoration without Clean Data,''
        in \emph{International Conference on Machine Learning}, 2018, pp. 2965--2974.

        \bibitem{zhang2017beyond}
        K. Zhang, W. Zuo, Y. Chen, D. Meng, and L. Zhang,
        ''Beyond a Gaussian Denoiser: Residual Learning of Deep CNN for Image Denoising,''
        \emph{IEEE Transactions on Image Processing}, vol. 26, no. 7, pp. 3142--3155, 2017.

        \bibitem{lefkimmiatis2018universal}
        S. Lefkimmiatis,
        ''Universal Denoising Networks: A Novel CNN Architecture for Image Denoising,''
        \emph{IEEE Transactions on Image Processing}, vol. 27, no. 2, pp. 919--933, 2018.

        \bibitem{zhang2018learning}
        K. Zhang, W. Zuo, Y. Chen, D. Meng, and L. Zhang,
        ''Learning Deep CNN Denoiser Prior for Image Restoration,''
        in \emph{IEEE Conference on Computer Vision and Pattern Recognition}, 2018, pp. 2808--2817.

        \bibitem{tai2017image}
        Y. Tai, J. Yang, and X. Liu,
        ''Image Super-Resolution via Deep Recursive Residual Network,''
        in \emph{IEEE Conference on Computer Vision and Pattern Recognition}, 2017, pp. 2790--2798.


	\end{thebibliography}
} % end \ifboolexpr
%
% for use as JACoW template the inclusion of the ANNEX parts have been commented out
% to generate the complete documentation please remove the "%" of the next two commands
%
%%%\newpage

%%%\include{annexes-A4}

\end{document} 